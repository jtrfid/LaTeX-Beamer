\documentclass{article}

% CJK packages
\usepackage{xeCJK}
\setCJKmainfont{SimHei}

\usepackage{tikz}
\usetikzlibrary{mindmap,trees}

%\usepackage[xetex,active,floats]{preview}

%\setlength\PreviewBorder{0mm}
%\PreviewEnvironment{tikzpicture}

\begin{document}
\pagestyle{empty}

\begin{tikzpicture}[remember picture, overlay]

  \node [shift={(10 cm,-7cm)}]  at (current page.north west)
  {%
        \begin{tikzpicture}[remember picture, overlay]
  
  \path[mindmap,concept color=blue!50!black,text=white]
    node[concept] {人工神经网络漫谈}
    [clockwise from=0]
    child[concept color=green!50!black] {
      node[concept] {1. 什么是ANN(What)}
      [clockwise from=90]
      child[concept color=orange] { node[concept] {概念} }
      child[concept color=orange] { node[concept] {特点} }
      child[concept color=orange] { node[concept] {功能} }
      child[concept color=orange] { node[concept] {应用领域} }
    }  
    child[concept color=green!50!black] {
      node[concept] {2. 如何构造ANN(How)}
      [clockwise from=-30]
      child[concept color=orange] { node[concept] {三要素} }
    }
    child[concept color=green!50!black] { node[concept] {3. M-P模型} }
    child[concept color=green!50!black] { node[concept] {4. 感知器\\ \& \\BP网络}
      [clockwise from=-135]
      child[concept color=orange] { node[concept] {BP网络} }
      child[concept color=orange] { node[concept] {ANN发展史} }
      child[concept color=orange] { node[concept] {感知器} }
    }
    child[concept color=green!50!black] { node[concept] {5. 其他网络综述} }
    child[concept color=green!50!black] { node[concept] {6. 实例(Case)} };
  \end{tikzpicture}
  
};

\end{tikzpicture}


\end{document}

%%% Local Variables: 
%%% mode: latex
%%% TeX-master: t
%%% End: 
